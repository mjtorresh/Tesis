\section{Introducción}
Diferentes contextos a lo largo de la historia han construido concepciones normativas de feminidad y de masculinidad. Esto ha generado que existan definiciones sociales de identidades femeninas y masculinas, y que ciertas acciones y comportamientos sean entendidos como femeninos y otros como  masculinos \citep{scott2007gender}. Cuando una persona realiza una acción entendida como femenina, se dice que tiene una expresión de género femenina. Las expresiones de género están asociadas con el desarrollo histórico de normas sociales sobre cómo debería actuar una persona de identidad femenina y una de identidad masculina. A pesar de que existen estas normas sociales, si la identidad es la concepción de uno mismo, en las interacciones sociales los agentes observan las acciones de los demás, mas no su identidad. Entonces, las expresiones de género señalizan la identidad y, por ello, pueden ser percibidas de diferentes maneras.

Este trabajo estudia cómo el observar diferentes expresiones de género afecta la disposición a interactuar entre pares. Que la disposición a interactuar con una persona cambie dependiendo las expresiones de género de esa persona, tiene implicaciones en el corto y largo plazo. En el corto plazo, limitar las interacciones con un grupo poblacional puede segmentar la sociedad. En sociedades segmentadas el intercambio y la transferencia de ideas es limitado; lo que puede generar un acceso desigual a los recurso. En el largo plazo, dado que actuar de manera diferente a lo socialmente prescrito genera una externalidad sobre los pares y es castigado, puede condicionar a las personas a actuar acorde a la norma social. Esto, a su vez, genera sociedades en las que se observan comportamientos estándar aun ante preferencias heterogéneas \citep{bernheim1994theory,akerlof2000economics}.

Con el fin de estudiar el impacto que tiene observar diferentes expresiones de género sobre la disposición de un agente a interactuar con otro, se realizó un expreimneto de campo en el que los participantes elegían a sus pares para el desarrollo de una tarea. El experimento replica el modelo de señalización que propone este estudio. Este modelo está basado en el propuesto por \cite{akerlof2000economics} pero toma la identidad de uno de los agentes como información privada; donde la identidad puede ser de tres tipos: masculina, femenina u otra. El agente con información privada decide señalizar su identidad con una expresión de género femenina o una masculina. El otro agente observa la señal, forma creencias sobre la identidad del otro y decide si interactúa o no con él. 

Del modelo se derivan dos condiciones para llegar a un equilibrio que sea interactuar ante cualquier expresión de género, cuando la acción que toma el agente con información privada es diferente dependiendo de cuál sea su identidad. La primera condición es que el agente con identidad debe cumplir la norma social. Es decir, si es de identidad femenina debe tener una expresión femenina y si es de identidad masculina debe tener una expresión masculina. Esta condición depende, a su vez, de la utilidad intrínseca que le genera a ese agente tomar una expresión femenina y la que le genera una expresión masculina, y la desutilidad de ir en contra de lo socialmente prescrito. La segunda condición es que el costo esperado de ver al otro violar la norma social debe ser menor al costo de no interactuar. 

El experimento consistió en un concurso de \textit{memes} en grupos de tres personas. Los participantes, primero, se inscribieron al concurso y, con la inscripción, reportaron sus expresiones de género. Luego, cada participante recibió una caracterización de otros ocho participantes. Dicha caracterización incluía unas expresiones de género y unas características generales. A partir de esa información, cada participante organizó los ocho perfiles según con quién prefería hacer el \textit{meme}. 

Los resultados sugieren que el cumplimiento de la norma social sí aumenta la disposición de otros a interactuar. En particular la evidencia sugiere que hay más disposición a interactuar con pares que tienen vestimenta femenina cuando reportan una habilidad femenina que cuando reportan una habilidad masculina. Por el contrario, la disposición a interactuar con pares que tienen vestimenta masculina es prácticamente independiente a la habilidad que reportan. 

Este trabajo se relaciona con la literatura existente de diferentes ramas. Primero, se basa en trabajos que han estudiado que la utilidad depende de la identidad y la adherencia a la norma social \citep{akerlof2000economics,bernheim1994theory,fang2005dysfunctional}; y la literatura que ha estudiado la existencia y transferencia de normas sociales de género \citep{giuliano2017gender, lundborg2017childrencareer, bertrand2015genderandincome, nollenberger2016mathgap, kleven2018children}. Este trabajo representa un contribución a esa literatura porque parte de la existencia de normas sociales de género y de las acciones que toman los agentes para estudiar si existe entre pares un castigo social como mecanismo de transmisión de esas normas y como mecanismo que genera conformismo. También aporta a esa literatura por enfocarse en normas sociales de género femeninas y masculina, no solo en las femeninas. 

Segundo, este trabajo contribuye a la literatura que ha utilizado experimentos para identificar patrones de discriminación \citep{cardenas2008discrimination, beautifulorwhite2012, howyoulookorspeak, mobius2006whybeautymatters, hoffdiscriminationsocialidentity}. Este trabajo contribuye a esta literatura al estudiar cómo otro grupo de características de las personas genera cambios en la manera que otros responden a ellos. También aporta a la anterior literatura al tomar como categoría de análisis los comportamientos y las acciones de los agentes, por encima de si la persona es mujer u hombre. De esta manera, este trabajo se desliga de las diferencias biológicas y se enfoca en el género como construcción social.

Lo que resta de este trabajo se divide de la siguiente manera: la sección 2 presenta el marco teórico; la sección 3 presenta el diseño experimental; la sección 4 describe los datos; la sección 5 presenta las hipótesis y los resultados; la sección 6 discute las implicaciones de los resultados y la sección 7 concluye. 
