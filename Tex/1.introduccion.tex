\section{Introducción}
Diferentes contextos a lo largo de la historia han construido concepciones normativa de feminidad y de masculinidad. Esto ha generado que existan definiciones sociales de identidades femeninas y masculinas, y que ciertas acciones y comportamientos sean entendidos como femeninos y otros como  masculinos \citep{scott2007gender}. Cuando una persona realiza una acción entendida como femenina, se dice que tiene una expresión de género femenina. Las expresiones de género están asociadas con el desarrollo histórico de normas sociales sobre cómo debería actuar una persona de identidad femenina y una de identidad masculina. A pesar de que existen estas normas sociales, si la identidad es la concepción de uno mismo, en las interacciones sociales los agentes observan las acciones de los demás, mas no su identidad. Entonces, las expresiones de género señalizan la identidad y, por ello, pueden ser percibidas de diferentes maneras.

Este trabajo estudia cómo observar diferentes expresiones de género afecta la disposición a interactuar entre pares. Si la disposición a interactuar con una persona cambia dependiendo del conjunto de expresiones de esa persona, las interacciones sociales se pueden convertir en un mecanismo a través del cual las normas sociales de género persisten. Esto, debido a que las normas sociales persisten cuando los agentes están condicionados a actuar según estas normas y cuando existe un control social para hacerlas cumplir \citep{epstein2006similarity}. Consistente con esto, trabajos anteriores han establecido que actuar de manera diferente a lo socialmente prescrito genera una externalidad sobre los pares, y es castigado y censurado por la sociedad \citep{bernheim1994theory,akerlof2000economics}. Este trabajo explora si una personas está más dispuesta interactuar con sus pares cuando estos señalizan que cumplen la norma social. 

Para estudiar el impacto de observar una serie de expresiones de género sobre la disposición de un agente a interactuar con otro, este trabajo consiste de un experimento de campo en el que los participantes eligen a sus pares para desarrollar una tarea. El experimento replica el modelo de señalización que propone este estudio. Este modelo está basado en el propuesto por \cite{akerlof2000economics} pero toma la identidad de uno de los agentes como información privada; donde la identidad puede ser de tres tipos: masculina, femenina u otra. El agente con información privada decide señalizar su identidad con una expresión de género femenina o una masculina. El otro agente observa la señal, forma creencias sobre la identidad del otro y decide si interactúa o no con él. 

Del modelo se derivan dos condiciones para llegar a un equilibrio que sea interactuar ante cualquier expresión de género, cuando la acción que toma el agente con información privada es diferente dependiendo de cuál sea su identidad. La primera condición es que el agente con identidad debe cumplir la norma social. Es decir, si es de identidad femenina debe tener una expresión femenina y si es de identidad masculina debe tener una expresión masculina. Esta condición depende a su vez de la utilidad intrínseca que le genera a ese agente tomar una expresión femenina y la que le genera una expresión masculina, y la desutilidad de ir en contra de lo socialmente prescrito. La segunda condición es que la creencias del agente con información incompleta sobre si el otro agente está violando la norma social debe ser menor al costo de no interactuar. 

En el experimento los participantes debían hacer una tarea en grupos de tres personas. Cada participante recibió una caracterización de otros ocho participantes. La información que recibieron sobre los otros participantes incluía algunas de sus expresiones de género. A partir de esas características, cada participante organizó los perfiles en el orden en el que quería que esa persona hiciera parte de su grupo para hacer la tarea. 

Los resultados sugieren que el cumplimiento de la norma social sí aumenta la disposición de otros a interactuar. En particular la evidencia sugiere que hay más disposición a interactuar con pares que tienen vestimenta femenina cuando reportan una habilidad femenina a cuando reportan una habilidad masculina. Por el contrario, la disposición a interactuar con pares que tienen vestimenta masculina es prácticamente independiente a la habilidad que reporta. Estas respuestas en la disposición a interactuar con otros son más fuertes entre las personas que se adhieren a la norma social con sus expresiones.  

Este trabajo se relaciona con la literatura existente de diferentes ramas. Primero, se basa en trabajos que han estudiado la que la identidad afecta el comportamiento de agentes económicos. Esto estudios han determinado  que la utilidad de los agentes no solo depende de las preferencias sobre las acciones sino también de su identidad y la adherencia a una norma social \citep{akerlof2000economics}. Esto puede llevar a que los agentes tomen acciones aceptadas socialmente para señalizar que sus preferencias están alineadas con la norma social y, por lo tanto, a la existencia de comportamientos estándares aun ante preferencias heterogéneas. Que la utilidad dependa de la identidad y la adherencia a la norma social se traduce en un conformismo por la norma social porque la aceptación social o el \textit{status} genera utilidad \citep{bernheim1994theory}. Este trabajo, más que entender si las personas se conforman con la norma social, parte de las acciones que tomas los agentes para analizar si existe entre pares el canal de castigo social que genera ese conformismo. De esta manera, une la literatura que ha explicado que la identidad afecta el comportamiento económico con la que ha estudiado que la identidad afecta la posibilidad de diversificar el riesgo a través de las interacciones \citep{fang2005dysfunctional}. Este trabajo representa una contribución a la literatura de identidad al estudiar, para el caso particular de la identidad de género,  la disposición a interactuar entre pares como canal que afecta la diversificación del riesgo y  genera conformismo.

Segundo, este trabajo contribuye a la literatura que ha estudiado la existencia de normas sociales de género. Trabajos recientes han identificado que condiciones de largo plazo, como las técnicas agrícolas iniciales, la religión, la estructura familiar y el lenguaje, están asociadas con el origen de normas y diferencias de género. Estos trabajos también han identificado que incluso cuando las condiciones asociadas al origen de algunas normas de género desaparecen, esas normas persisten y explican actuales diferencias de género \citep{giuliano2017gender}. Otros trabajo se han enfocado en identificar diferencias actuales de género asociadas a normas sociales \citep{lundborg2017childrencareer, bertrand2015genderandincome} y estudian la trasmisión intergeneracional como mecanismo de persistencia de esas normas \citep{nollenberger2016mathgap, kleven2018children}. Este trabajo representa una contribución a esa literatura porque parte de la existencia de normas sociales de género para estudiar las interacciones de pares como mecanismo de trasmisión de estas normas. También aporta a esta rama de la literatura por enfocarse en normas sociales de género femeninas y masculinas, no solo en las femeninas.

Tercero, este trabajo contribuye a la literatura de economía del comportamiento que ha utilizado experimentos para identificar patrones de discriminación. Estos trabajos han identificado que existen características de las personas que generan un cambio en el comportamiento de otros hacia ellos en la elección de grupos o la distribución de recursos  \citep{cardenas2008discrimination, beautifulorwhite2012, howyoulookorspeak, mobius2006whybeautymatters}. También han identificado que recordarle a las personas una característica suya que ha sido un factores de discriminación o está asociada a un estereotipo prima sus propias acciones  \citep{hoffdiscriminationsocialidentity}. Este trabajo contribuye a esta literatura al estudiar cómo otro grupo de características de las personas genera cambios en la manera que otros responden a ellos. 

Este trabajo también se relaciona con la literatura que se ha enfocado en identificar diferencias en comportamiento entre mujeres y hombres en aversión al riesgo, competitividad, reciprocidad, confianza, y altruismo, entre otros \citep{gneezy2009gender, cardenas2012gender, croson2009gender, solnick2001genderultimatumgame, eckel2001chivalryultimatumgame, gomez2018gendernegociacion, buchan2008trustandgender, espinosa2015prosocialandgender}.\footnote{Esta literatura ha encontrado efectos mixtos en la reciprocidad, la confianza y el altruismo; y ha encontrado que las mujeres suelen ser más aversas al riesgo y menos competitivas que los hombres. Estas diferencias ha sido presentada como un hecho estilizado, sin embargo para la comunidad Khasi, que sigue una linea matriarcal, se observa que las mujeres son más competitivas que los hombres \citep{gneezy2009gender}. Tampoco se cumplió ese hecho entre niños y niñas colombianas \citep{cardenas2012gender}. Por lo que, es difícil argumentar que por diferencias biológicas entre mujeres y hombres existen diferencias en el comportamiento. Incluso en los trabajos que se enfocan en el efecto del estrógeno y la testosterona sobre el comportamiento no se ha podido determinar el papel que juega la biología en el competitividad y en la aversión al riesgo \citep{sapienza2009genderriskaversion,van2012testosteroneandcoperation,zethraeus2009rctestrogentestosterone,eisenegger2010testosteroneandbargainign}.}
Estos trabajos al centrarse en diferencias entre hombres y mujeres dejan a un lado las heterogeneidades que existen entre los mismos. Además, estos trabajo, en su mayoría, presentan esas diferencias entre hombres y mujeres como el resultado de diferencias intrínsecas, sin considerar las normas sociales que operan para que existan esas diferencias. Este trabajo aporta a la anterior literatura al tomar como categoría de análisis los comportamientos y las acciones de los agentes por encima de si la persona es mujer u hombre. De esta manera, este trabajo se desliga de las diferencias biológicas y se enfoca en el género como construcción social.

Lo que resta de este trabajo se divide de la siguiente manera. La sección 2 presenta el marco teórico. La sección 3 presenta el diseño experimental. La sección 4 describe los datos. La sección 5 presenta las hipótesis y los resultados, y la sección 6 concluye. 
