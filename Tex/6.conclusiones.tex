\section{Conclusiones}
Este trabajo estudia de qué manera observar diferentes expresiones de género, cuando la identidad de género es información privada, cambia la disposición a interactuar entre pares. Para esto, este trabajo primero presenta un juego de información incompleta en el que la identidad de uno de los jugadores es información privada. Ese jugador conoce su identidad y debe elegir entre tener un expresión de género femenina o una masculina. El otro jugador observa la expresión de género de ese agente, a partir de esto forma creencias sobre la identidad del otro jugador y decide si interactúa o no con él. Segundo, este trabajo incluye un experimento de campo en el que los participantes debían elegir con quién desarrollar una tarea, únicamente con la información de unas de sus expresiones de género, la edad y el departamento de nacimiento. 

Los resultados del modelo sugieren que existen dos condiciones principales para que los agentes lleguen a un equilibrio en el que interactúen y en el que el jugador con información privada tome una acciones diferente dependiendo de su identidad (tal como sucede en los datos). Primero, si jugador con información privada es de identidad femenina o de identidad masculina, debe adherirse a la norma social. Es decir, debe tener una expresión femenina o masculina, respectivamente. Segundo, el costo de no interactuar debe ser mayor o igual al costo esperado de que el otro jugador esté violando la norma social. 

Los resultados experimentales sugieren que la disposición a interactuar entre pares sí cambia cuando cambia la combinación de expresiones de género que estos observan de los demás. Principalmente hay una baja disposición a interactuar con pares que presentan una vestimenta muy femenina y que reportan ser más hábiles en matemáticas que comunicándose. Los resultados también sugieren que entre las vestimentas más masculinas la disposición de sus pares a interactuar con ellos no está determinada por la habilidad. Mientras que entre las vestimentas más femeninas sí hay un beneficio de reportar una habilidad femenina comparado a reportar una habilidad masculina. 

Los resultados sugieren que la baja disposición a interactuar con personas que presenta una vestimenta femenina y que son más hábiles en matemáticas está explicado por la elección de los participantes de identidad femenina. Mientras que la alta preferencia por trabajar con pares que tienen vestimenta femenina y son hábiles comunicándose está explicada principalmente por el comportamiento de los participantes que cumplen la norma social con sus expresiones; y en menor medida por el comportamiento de participantes de identidad masculina. 

Por último, los resultados del experimento sugieren que las personas que no cumplen la norma social con sus expresiones de género, eligen con quién de desarrollar la tarea considerando su habilidad y su vestimenta independientemente. Sin embargo, la elección de estos últimos se traduce en una menor disposición a interactuar con personas que tienen vestimenta femenina. 

El hecho de que el cumplimiento de las normas sociales cambie la disposición que tienen los demás a interactuar, tiene varias implicaciones. Una de ellas es que genera ineficiencias. Bajo el supuesto de que hay ganancias de interactuar con otros, dejar de hacerlo porque una persona no cumpla una norma social o por un prejuicio implica perder esas ganancias. Por lo tanto, las normas sociales implican un dilema social tanto porque genera que los agentes tengan comportamiento homogéneos aunque no sea consistente con sus preferencias, como porque trunca el desarrollo de redes sociales, que son un mecanismo para mitigar diversos riesgos, el desarrollo de instituciones y el desarrollo económico de una sociedad. 

Una limitación importante de este trabajo es que no mide si los participantes asociación la vestimenta, las habilidades y las aspiraciones al género. Futuros estudios podrían medirse esta asociación con una prueba de asociación implícita. A pesar de que este trabajo tiene esa limitación, la distribución de expresiones de género por identidades sugiere que entre los participantes del experimento sí existe la asociación para la vestimenta y la habilidad. 

Otra limitación de este trabajo es que en el experimento, los participantes podían interactuar por correo o celular sin necesidad de encontrarse físicamente. Esto genera que los resultados sean una cuota inferior de la disposición a interactuar. En la encuesta de salida solo el 17\% reporta haberse reunido con su grupo para hacer la tarea. Estudios futuros podrían cambiar el marco del concurso para aumenta la probabilidad de que los participantes interactúen frente a frente. 

Este trabajo también tiene la limitación de que los resultados son sugestivos. Debido a que el tamaño de la muestra es pequeño y por lo que no tiene poder para identificar efectos estadísticamente significativos. Además, el experimento tiene la limitación de que hubo una atrición del 38\% entre la inscripción y el proceso de elección. Esto podría corregirse en futuros estudios recogiendo más información de contacto en la inscripción y acortando los tiempo de las diferentes etapas. 