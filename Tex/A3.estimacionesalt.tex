\section{Estimaciones alternativas}
\begin{table}[ht!]
    \centering
    \caption{Logit rankeado y ordenado}
    \label{tab:rol}
    \begin{threeparttable} \fontsize{8.5}{12}\selectfont {
    \begin{tabular}{lccc} \hline \hline
                                                & \multicolumn{3}{c}{Ranker}        \\\cmidrule{2-4}
                                                &   Todos   &  Femenino & Masculino \\ 
                                                &   (1)     &    (2)    &   (3)     \\ \hline
                                                &           &           &           \\
    Puntaje vestimenta  	                    &	-0.111	&	-0.280*	&	0.006	\\
                                        	    &  	(0.112)	&	(0.164)	&	(0.153)	\\
    Habilidad comunicación  	                &	0.225	&	0.394*	&	0.058	\\
    	                                        &	(0.151)	&	(0.234)	&	(0.190)	\\
    Habilidad comunicación*Puntaje vestimenta 	&	0.132	&	0.246	&	0.100	\\
                                            	&	(0.124)	&	(0.193)	&	(0.176)	\\
    Aspiración familiar 	                    &	0.141	&	0.339*	&	-0.024	\\
    	                                        &	(0.142)	&	(0.203)	&	(0.216)	\\
    Edad	                                    &	-0.026	&	-0.049	&	-0.015	\\
    	                                        &	(0.038)	&	(0.056)	&	(0.056)	\\
    Bogotá = 1	                                &	-0.032	&	-0.035	&	-0.041	\\
                                            	&	(0.146)	&	(0.242)	&	(0.180)	\\
    Posición Inicial	                        &. 0.156*** & 0.171***	& 0.146***	\\
    	                                        &	(0.031)	&	(0.044)	&	(0.044)	\\

                                                &           &           &           \\
    Observaciones                               &   560     &   264     &   296     \\
    Número de rankers                           &   70      &   33      &   37      \\ \hline \hline
    \end{tabular}}
    \begin{tablenotes}
    \footnotesize{
    \item \textit{Nota:} Estimaciones por logit rankeado y ordenado del efecto de las expresiones de género en el ranking que recibieron por parte de sus pares. Puntaje vestimenta es el puntaje estandarizado de qué tan femenina se percibe la vestimenta que se observa. Habilidad comunicación es el indicador que toma el valor de uno si se observa que el par es más hábil comunicándose que en matemáticas. Aspiración familiar es el indicador que toma el valor de uno si se observa que el par aspira más a formar familia que a tener comodidad económica. Edad es la edad del par que se observa. Bogotá es un indicador que toma el valor de uno si el par que se observa nació en Bogotá, y posición inicial es la posición que ocupaba el par en el listado antes de que los perfiles fueran organizados. 
    \item Errores estándares robustos en paréntesis; *** p$<$0.01, ** p$<$0.05, * p$<$0.1.}
    \end{tablenotes}
    \end{threeparttable}
\end{table}

\begin{table}[ht!]
    \centering
    \caption{Ranking promedio}
    \label{tab:mean}
    \begin{threeparttable} \fontsize{8.5}{12}\selectfont {
    \begin{tabular}{lccc} \hline \hline
                                                & \multicolumn{3}{c}{Ranker}        \\\cmidrule{2-4}
                                                &   Todos   &  Femenino & Masculino \\ 
                                                &   (1)     &    (2)    &   (3)     \\ \hline
                                                &           &           &           \\
    Puntaje vestimenta                      	&	-0.101	&	-0.234	&	0.034	\\
                                            	&	(0.168)	&	(0.283)	&	(0.242)	\\
    Habilidad comunicación  	                &	0.479**	&	0.470	&	0.332	\\
    	                                        &	(0.209)	&	(0.321)	&	(0.308)	\\
    Habilidad comunicación*Puntaje vestimenta 	&	0.141	&	0.189	&	-0.026	\\
                                            	&	(0.232)	&	(0.349)	&	(0.328)	\\
    Aspiración familiar 	                    &	0.196	&	0.600*	&	0.006	\\
                                            	&	(0.206)	&	(0.310)	&	(0.303)	\\
    Edad	                                    &	0.001	&	-0.006	&	0.031	\\
                                            	&	(0.061)	&	(0.100)	&	(0.102)	\\
    Bogotá = 1	                                &	-0.045	&	-0.115	&	-0.086	\\
    	                                        &	(0.212)	&	(0.328)	&	(0.304)	\\
    Posición Inicial	                        &	0.264**	& 0.367***	&	0.237*	\\
    	                                        &	(0.123)	&	(0.107)	&	(0.127)	\\
    Constante	                                &	2.933*	&	2.422	&	2.710	\\
    	                                        &	(1.561)	&	(2.082)	&	(2.330)	\\
    	                                        &		    &		    &		    \\
    Observaciones	                            &	113	    &	110	    &	112	    \\\hline \hline
    \end{tabular}}
    \begin{tablenotes}
    \footnotesize{
    \item \textit{Nota:} Estimaciones por mínimos cuadrados ordinarios del efecto de las expresiones de género en el promedio de rankings. Puntaje vestimenta es el puntaje estandarizado de qué tan femenina se percibe la vestimenta que se observa. Habilidad comunicación es el indicador que toma el valor de uno si se observa que el par es más hábil comunicándose que en matemáticas. Aspiración familiar es el indicador que toma el valor de uno si se observa que el par aspira más a formar familia que a tener comodidad económica. Edad es la edad del par que se observa. Bogotá es un indicador que toma el valor de uno si el par que se observa nació en Bogotá, y posición inicial es el promedio de las posiciones que ocupaba el par en los listados antes de que los perfiles fueran organizados. 
    \item Errores estándares robustos en paréntesis; *** p$<$0.01, ** p$<$0.05, * p$<$0.1.}
    \end{tablenotes}
    \end{threeparttable}
\end{table}
