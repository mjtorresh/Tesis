\section{Protocolo experimental}
\subsection{Inscripción y Encuesta}
\textit{Dos semanas: del 20 al 31 de enero.\\
Reclutamiento con afiches, volantes y a través de redes sociales.\\}
Agradecemos su participación en este ejercicio que hace parte de un estudio sobre decisiones individuales y bienestar universitario. Usted podrá ganar hasta \$70.000 pesos dependiendo de las decisiones que tomen usted y otras personas de la universidad. Usted puede retirarse en cualquier momento. Pero, solamente si completa el ejercicio, recibirá la cantidad de dinero que gane por su desempeño. \\
El dinero utilizado para pagarle a los participantes ha sido aportado por financiadores externos al estudio.\\

Este ejercicio consiste en un concurso de memes sobre bienestar universitario. Los memes deben comunicar un problema o una solución a un problema de bienestar universitario y deben ser desarrollados en grupos de tres personas. Un panel de jurados\footnote{El panel de jurados está conformado por profesores y asistentes de investigación de la Facultad de Economía de la Universidad de los Andes.} elegirá los 7 mejores memes y los investigadores entregarán los premios correspondientes a los grupos que hayan presentado esos memes. Los premios son:

\begin{itemize}
    \item Primer puesto \$210.000
    \item Segundo puesto \$180.000
    \item Tercer puesto \$150.000
    \item Cuarto puesto \$120.000
    \item Quinto puesto \$90.000
    \item Sexto puesto \$60.000
    \item Séptimo puesto \$30.000
\end{itemize}
El concurso tiene tres partes: inscripción, conformación de grupos y desarrollo de memes.\footnote{La inscripción está abierta del 20 al 31 de enero. Los grupos se conformarán del 3 al 7 de febrero y los memes se desarrollarán del 10 al 20 de febrero. El 28 de febrero se anuncian los ganadores.}\\
A continuación, le vamos a explicar cómo se conforman los grupos. \textbf{Por favor lea con atención las siguientes instrucciones.}\\
Los grupos se determinarán siguiendo el siguiente proceso:
\begin{itemize}
    \item Los investigadores conformarán secciones de 9 personas para formar 3 grupos de 3 personas.
    \item Usted observará una tarjeta con características generales para cada una de las otras 8 personas de su sección.
    \item Usted deberá organizar esas 8 tarjetas en el orden en el que quiere que hagan parte de su grupo.
    \begin{itemize}
        \item La persona con la que más le gustaría desarrollar el meme, debería ponerla en el primer lugar.
        \item La segunda persona con la que usted más le gustaría desarrollar el meme, debería ponerla en el segundo lugar.
        \item De tal manera que, la persona con la que menos le gustaría desarrollar el meme, debería ponerla en el último lugar.
    \end{itemize}
\end{itemize}
\textit{Grupo 1}
\begin{itemize}
    \item Luego de recibir todas las listas, los investigadores elegirán al azar a una de las nueve personas de la sección. Llamemos esa persona, la persona A.
    \item El grupo 1 será conformado por la persona A y las dos primeras personas en la lista de la persona A.
\end{itemize}
\textit{Grupo 2}
\begin{itemize}
    \item De las seis personas que no fueron asignadas al grupo 1, los investigadores elegirá una persona al azar. Llamemos a esa persona, la persona B.
    \item El segundo grupo será conformado por la persona B y las dos personas que ocupen los primeros puestos de la lista de la persona B (que no hayan sido asignados al grupo 1).
\end{itemize}
\textit{Grupo 3}
\begin{itemize}
    \item El tercer grupo será conformado por las tres personas que no fueron asignadas ni al grupo 1, ni al 2.
\end{itemize}
Para el desarrollo de esta actividad por favor respondan la siguiente encuesta que nos permitirá identificar cuáles son los temas de bienestar universitario que más le conciernen y conocer información básica suya.
Toda la información que usted brinde en este estudio es confidencial y será utilizada solamente con fines académicos. 
Ni su nombre, ni su identificación aparecerán en los informes o reportes de este estudio.\\
Dada esta información, queremos confirmar que usted desea participar en este estudio. Acceda al consentimiento informado 
\href{https://www.dropbox.com/s/2l9pwtj3sp1vbax/Consentimiento\%20Informado.pdf?dl=0}{acá}.\\

\pagebreak

\subsubsection{Encuesta de entrada}
\begin{enumerate}
\item Nombre y Apellido:
\item Correo Uniandes (incluya @uniandes.edu.co):
\item Género: (Femenino, Masculino, Otro)
\item Edad: (Min 18)
\item Departamento de nacimiento:
\item Programa principal: (lista)
\item Doble programa: (lista)
\item ¿Se ha cambiado de carrera?
\item ¿Cuántos semestres ha cursado? (Incluya el semeste que está cursando):
\item ¿Cuál es el nivel educativo más alto alcanzado por su madre? (lista)
\item ¿Cuál es el nivel educativo más alto alcanzado por su padre? (lista)
\item ¿Cuántos hermanos tiene? (Si no tiene hermanos, responda 0):
\item Seleccioné cuál de las siguientes vestimentas usaría usted con mayor frecuencia para ir a la universidad: (Imágenes de vestimentas, en orden aleatorio)
\item Para cada una de las siguientes, indique qué tan hábil es usted. Donde 1 indica no es hábil y 4 indica es fuertemente hábil.
\begin{enumerate}
    \item Comunicación (escucha, habla y escritura):
    \item Análisis e investigación:
    \item Adaptabilidad:
    \item Matemáticas:
    \item Tomar decisiones:
    \item Planeación:
    \item Hablar en público:
    \item Programación:
\end{enumerate}
\item Elija la opción que mejor describa sus habilidades
\begin{enumerate}
    \item Mucho más hábil en matemáticas | Algo más hábil en matemáticas | Algo más hábil comunicándose |  Mucho más hábil comunicándose
    \item Mucho más hábil adaptándose a diferentes escenarios | Algo más hábil adaptándose a diferentes escenarios | Algo más hábil en planeación | Mucho más hábil en planeación
\end{enumerate}
\item \textbf{Recuerde que este concurso de memes es sobre alguno de los siguientes problemas de bienestar universitario.} Para cada uno de los siguientes problemas de bienestar universitario indique el grado de importancia que usted les da. Donde 1 es no es importante y 4 es muy importante.
\begin{enumerate}
\item Carga académica:
\item Manejo de estrés por parte de los estudiantes:
\item Exclusión social:
\item Dificultad para expresar opiniones e ideas:
\item Falta de sentido:
\item Poca atención de los profesores a los estudiantes:
\item Falta de baños unificados
\item Soledad:
\item Dificultades económicas:
\item Exceso de competencia e individualismo entre estudiantes:
\item Plagio:
\item Falta de apoyo psicológico:
\item Excesiva atención a las notas:
\item Dificultades para trabajar en grupo:
\item Consumo de alcohol y drogas:
\end{enumerate}
\item Para cada una de las siguientes, indique de 1 a 4 qué tan importante es para usted tener \_\_\_\_\_\_\_ en los próximos 10 años. Donde 1 indica no es importante y 4 indica es esencial.
\begin{enumerate}
\item Estabilidad familiar
\item Una maestría
\item Un doctorado
\item Vivir muy cómodo económicamente
\item Reconocimiento público 
\item Vida espiritual
\item Bienestar físico
\end{enumerate}
\item Elija la opción que mejor describa sus aspiraciones para los próximos 10 años
\begin{enumerate}
    \item Aspira mucho más tener reconocimiento público | Aspira algo más tener reconocimiento público | Aspira algo más tener vida espiritual |  Aspira mucho más tener vida espiritual
    \item Aspira mucho más tener comodidad económica | Aspira algo más tener comodidad económica | Aspira algo más tener una familia estable	Aspira mucho más tener una familia estable
\end{enumerate}
\item ¿Ha contado con apoyo financiero durante su carrera?
\item Desde que entró a la universidad, ¿ha asistido a consejería académica o psicológica?
\end{enumerate}
Gracias por participar en este ejercicio, ha terminado la primera parte\\\
El 3 de febrero usted recibirá un correo para que pueda elegir sus grupo.


\subsection{Formación de grupos}
\textit{Una semanas: del 3 al 7 de febrero.\\
Contacto a través del correo. \\}
\fbox{
\begin{minipage}[t]{1\textwidth}
\textit{Nombre,}\\
Gracias por participar en el concurso de memes sobre bienestar universitario.\\
Para elegir el grupo con el que va a realizar el meme, por favor ingrese \textbf{desde un computador} al siguiente link: \textit{link votación}. Su identificador personal para entrar a esa página es \textit{(ID asignado)}.\\

Cordial saludo,\\
Equipo concurso bienestar universitario
\end{minipage}
}
\vspace*{0.5cm}\\ 
Agradecemos su participación en este ejercicio que hace parte de un estudio sobre decisiones individuales y bienestar universitario. Recuerde que usted podrá ganar hasta \$70.000 pesos dependiendo de las decisiones que tomen usted y otras personas de la universidad. Usted puede retirarse en cualquier momento. Pero, solamente si completa el ejercicio, recibirá la cantidad de dinero que gane por su desempeño.\\

\noindent\textbf{\Large{La siguiente parte únicamente se puede completar en computador. Por favor no responda desde un celular o tableta.}}\\
Los grupos se determinarán siguiendo el siguiente proceso:
\begin{itemize}
    \item Los investigadores conformarán secciones de 9 personas para formar 3 grupos de 3 personas.
    \item Usted observará una tarjeta con características generales para cada una de las otras 8 personas de su sección.
    \item Usted deberá organizar esas 8 tarjetas en el orden en el que quiere que hagan parte de su grupo.
    \begin{itemize}
        \item La persona con la que más le gustaría desarrollar el meme, debería ponerla en el primer lugar.
        \item La segunda persona con la que usted más le gustaría desarrollar el meme, debería ponerla en el segundo lugar.
        \item De tal manera que, la persona con la que menos le gustaría desarrollar el meme, debería ponerla en el último lugar.
    \end{itemize}
\end{itemize}
\textit{Grupo 1}
\begin{itemize}
    \item Luego de recibir todas las listas, los investigadores elegirán al azar a una de las nueve personas de la sección. Llamemos esa persona, la persona A.
    \item El grupo 1 será conformado por la persona A y las dos primeras personas en la lista de la persona A.
\end{itemize}
\textit{Grupo 2}
\begin{itemize}
    \item De las seis personas que no fueron asignadas al grupo 1, los investigadores elegirá una persona al azar. Llamemos a esa persona, la persona B.
    \item El segundo grupo será conformado por la persona B y las dos personas que ocupen los primeros puestos de la lista de la persona B (que no hayan sido asignados al grupo 1).
\end{itemize}
\textit{Grupo 3}
\begin{itemize}
    \item El tercer grupo será conformado por las tres personas que no fueron asignadas ni al grupo 1, ni al 2.
\end{itemize}
A continuación usted observará los ocho perfiles de las personas de su sección y deberá organizarlos según como quiere que participen de su grupo. \textbf{Recuerde que el orden de su lista determinará quién estará en su grupo.}\\
\textbf{Por favor ordene los siguientes perfiles en el orden en el que quiere que hagan parte de su grupo.\\
Recuerde que el orden de su lista determinará quién estará en su grupo.}\\
\textcolor{red}{Insertar screenshot de orden.}\\
Para los siguientes grupos, elija con cuál preferiría desarrollar el meme para este concurso. Se presenta 4 veces la siguiente decisión:\\
\textcolor{red}{Insertar screenshot de decisión.}\\
Gracias por participar en este ejercicio, ha terminado la segunda parte.\\
El 10 de febrero usted recibirá un correo para informarle quiénes están en su grupo y el correo para que los pueda contactar.

\subsection{Desarrollo de actividad}
Su identificador de grupo para este concurso es (ID asignado). 

Los miembros de este grupo son:
\begin{enumerate}
    \item Integrante 1. Correo
    \item Integrante 2. Correo
    \item Integrante 3. Correo
\end{enumerate}
En este grupo deben desarrollar un meme en el que se presente una descripción de uno de los siguiente problema de bienestar universitario y se proponga una solución:
\begin{enumerate}
\item 
\item 
\end{enumerate}

El vídeo debe ser enviado antes del \textbf{fecha a las 23:59}. 
El día \textit{8 días después de la fecha de entrega} se conocerán los dos grupos ganadores. 

\subsection{Encuesta de cierre}
Gracias por participar en el concurso. 
Hemos recibido su meme. Para que su meme sea tomado en cuenta por el panel de jurados, debe completar siguiente encuesta que no le tomará más de 10 minutos.  
Por último queremos pedirle que evalúe de 0 a 100 el desempeño de los demás miembros de su grupo ingresando al siguiente hipervínculo: 
El día \textit{8 días después de la fecha de entrega} se conocerán los dos grupos ganadores.




















